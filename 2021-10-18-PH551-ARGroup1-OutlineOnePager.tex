\documentclass[journal, a4paper]{IEEEtran}
\usepackage{csquotes}
\usepackage{textgreek}
\usepackage{graphicx}
\usepackage{amsmath}
\usepackage{textcomp}
\usepackage{hyperref}
\usepackage{comment}
% Header and Footer

\usepackage{fancyhdr}
\pagestyle{fancy}
\fancyhead{}
\fancyfoot{}
\fancyfoot[L]{Kata Benedek, Christofer Iacovou, Lewis Russell}
\fancyfoot[R]{4th Yr Adv. Res. Group 1}
\fancyfoot[C]{}
\renewcommand{\headrulewidth}{1pt}
\renewcommand{\footrulewidth}{1pt}

% Your document starts here!
\begin{document}

\onecolumn
\setcounter{page}{1}

\begin{center}
\huge{\bfseries PH551 EPSRC Proposal Outline} \\
\textsc{\Large Mitigating Joule Expansion in Multicell Atomic Quantum Memory} 
\end{center}

{\fontfamily{cmr}
\large 

\noindent\makebox[\linewidth]{\rule{\textwidth}{0.4pt}}

%%%%%%%%%%%%%%%%%%%%%%%%%%%%%%%%%%%%%%%%%%%%

\vspace{0.3cm}

\small{The proposed research in this application seeks to merge two existing technologies in order to improve the memory lifespan of Multicell Atomic Quantum Memory (MAQM). By utilising optical tweezers to reduce the effects of atomic free expansion \cite{OptTweezer}, dipole trap arrays can house microensembles of 2D atomic memory cell arrays and increase the memory lifespan of existing MAQM technology \cite{MainMAQM}.}

%%%

\section{General Background}

\vspace{0.2cm}

The realisation of an effective method to read and write quantum information is of upmost importance to resolving current limitations in quantum communication systems \cite{qRAMwalk}. Before scalable large-scale quantum networks can become a reality, various hardware design proposals for how to store communicated quantum information must be explored. Quantum Computing and Communication offer several advantages over the Classical Systems used today - namely, their superior computational ability in certain tasks grants the capability to break current data encryption algorithms \cite{postQcrypto} and to offer greater computational disposal to future scientific research. Furthermore, they are unrivaled in their security of end-to-end information exchange \cite{PubKeyQcrypto}. 

However, with the physical absence of scalable quantum information storage designs, the search for appropriate solutions continues. One such method recently developed involved a quantum repeater node to fulfil such purposes \cite{MainMAQM}. Here, different segments of atom-photon entanglement where efficiently connected using their MAQM design, and the ability to address individual multimode storage states allowed for the demonstration of a Random Access Quantum Memory (RAQM, or qRAM) module. Memory lifespan was limited however, to a duration of around one millisecond - the main limitation of this design. Extending this time duration offers great potential for this qRAM design to be integrated into both current and future quantum communication systems.  

%%%

\section{Objectives}

\vspace{0.2cm}

- Ultimately, improve the memory lifespan of the MAQM design from \cite{MainMAQM}. 

- Draw on \cite{MainMAQM} and \cite{OptTweezer} to develop ``our new amazing design". Make-up some collaboration bullshit with \cite{MainMAQM} and/or \cite{OptTweezer}? :D

- Breakdown into four-year timeline?

- Have a new qRAM design that can be integrated into existing/future quantum systems.

%%%

\section{Methodology}

\vspace{0.2cm}

- Take the MAQM design from \cite{MainMAQM} where they use it as qRAM, then merge with the optical tweezers array method described in \cite{OptTweezer}. How we propose to do this still needs thinking. (basically, replacing the magneto-optical-trap (MOT) method used in \cite{MainMAQM} with the dipole trap method from \cite{OptTweezer}).

- Suggest how we create this experimental set-up.

- (I put this one in General Background, but belongs in Methodology if anywhere)...... ``In their magneto-optical-trap (MOT) method, used to contain the memory array of the MAQM, atoms are still subject to significant disruption due to atomic free expansion, where the atoms' desired for thermal equilibrium creates a tendency to diffuse from the desired 2D atomic memory array configuration.".

%%%%%%%%%%%%%%%%%%%%%%%%%%%%%%%%%%%%%%%%%%%%

\vfill

\bibliographystyle{unsrt}
\bibliography{2021-10-18-PH551-GroupI3-OutlineOnePager}
\vspace{-0.6cm}

}

\end{document}